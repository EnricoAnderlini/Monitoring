\documentclass{beamer}

\usetheme{Frankfurt}
\usecolortheme{dove}
\usepackage[utf8]{inputenc}
\usepackage[T1]{fontenc}
\usepackage{lmodern}
\usepackage{graphicx}
\usepackage{listings}
\usepackage{xcolor}

\usetheme{AnnArbor}
\usecolortheme{spruce}

\title[Reforestation Project]%
{Reforestation Project in the Pool region, South-East Congo}

\author{Enrico Anderlini}
\institute{University of Bologna - Unibo\\

\href{https://github.com/EnricoAnderlini}{Link to my GitHub account: https://github.com/EnricoAnderlini}}

\date{} % Leave empty if you don't want a date on the title page

\begin{document}

%--- First Slide (Title Page) -----------------------------------------%
\begin{frame}
  \titlepage
\end{frame}

%--- Second Slide: Dataset --------------------------------------------%
\begin{frame}{The Dataset: Landsat 2 level-2}
  \begin{itemize}
    \item \textbf{Downloaded from USGS Earth Explorer:} user-friendly platform where you can intuitively select any area of interest, and the data set needed
    \item \textbf{Landsat 2 Level-2 for vegetation studies:} high-resolution images, especially in the Red and Near-Infrared (NIR) bands, from 30m
  \end{itemize}
\end{frame}

%--- Third Slide: Area of study --------------%
\begin{frame}{Area of study}
 {\Large \textbf{Pool Region, Congo}}\\[0.5em]
  {\small National Reforestation programme (PRONAR), intended to plant 42.000 acacia trees }\\[2em]
\begin{figure}
      \centering
      \includegraphics[width=0.8\linewidth]{Congo pool.png}
      \label{fig:enter-label}
  \end{figure}
\end{frame}

%--- Third Slide: Data set --------------%
\begin{frame}[fragile]{Import data set}
  \textbf{Step-by-step approach for importing the data set:}
  \vspace{0.1cm}
  \scriptsize  
  \begin{verbatim}
  setwd("C:/Users/Utente/Desktop/Enrico/Università/Progetto Monitoring/Congo")
  library(terra)
  library(imageRy)
  library(ggplot2)
  library(viridis)
  
  # Load all bands from 30m for all the different years
  # example with the year 2023
  nir23 <- rast("Giugno 23/LC08_L2SP_182062_20230622_20230630_02_T1_SR_B5.TIF")
  red23 <- rast("Giugno 23/LC08_L2SP_182062_20230622_20230630_02_T1_SR_B4.TIF")
  green23 <- rast("Giugno23/LC08_L2SP_182062_20230622_20230630_02_T1_SR_B3.TIF")
  blue23 <- rast("Giugno 23/LC08_L2SP_182062_20230622_20230630_02_T1_SR_B2.TIF")

 \end{verbatim}
\end{frame}

%--- Fourth Slide: Natural colors 2013 --------------%
\begin{frame}{Natural colors 2013}
 \texttt{Congo2013 <- c(red13, green13, blue13, nir13)
im.plotRGB(Congo2013, r=1, g=2, b=3) \\}
  \begin{center}
    \includegraphics[width=0.4\textwidth]{Natural_colors2013.png}     
  \end{center}
\end{frame}

%--- Fifth Slide: Comparison of Three Months -------------------------%
\begin{frame}[fragile]{False colors 2013}
  \begin{itemize}
      \item  im.plotRGB(Congo2013, r=4, g=2, b=3) 
 \item im.plotRGB(Congo2013, r=3, g=2, b=4) 
 \item im.plotRGB(Congo2013, r=3, g=4, b=2)
  \end{itemize}
  \vspace{0.3cm}
   \begin{center}
    \vspace{0.5cm}
    \includegraphics[width=0.9\textwidth]{False_colors_2013.png} 
  \end{center}
\end{frame}

%--- Sixth Slide: Calculation of DVI -----------------------------------%
\begin{frame}[fragile]{Calculating the DVI Variations}
  \vspace{0.5cm}
    \textbf Using the NIR and Red channels to calculate the Difference Vegetation Index 
        
    \begin{itemize}
    \vspace{0.5cm}
   \item  dvi2013 = Congo2013[[4]] - Congo2013[[1]]
   \item  dvi2018 = Congo2018[[4]] - Congo2018[[1]]
   \item  dvi2023 = Congo2023[[4]] - Congo2023[[1]]     
    \end{itemize}
     \vspace{0.5cm}  
\end{frame}

%--- Seventh Slide: Boxplot of DVI -------------------------------------%
\begin{frame}[fragile]{DVI Variations over 10 years}
  \vspace{0.5cm}
  \textbf{Comparsion of the DVI between 2013, 2018 and 2023}
  \vspace{3.5cm}
  \centering
    \includegraphics[width=1.0\textwidth]{dvi2013_2018_2023.png}
 \end{frame}

 %--- Eighth Slide: Differences between years -------------------------------------%
\begin{frame}[fragile]{NDVI variations from 2013 to 2023}
  \textbf{Calulate the NDVI for the years 2013, 2018 and 2023}
  \vspace{0.5cm}
  \begin{itemize}
      \item ndvi2013 = dvi2013 / (Congo2013[[4]] + Congo2013[[1]])
      \item ndvi2018 = dvi2018 / (Congo2018[[4]] + Congo2018[[1]])
\item ndvi2023 = dvi2023 / (Congo2023[[4]] + Congo2023[[1]])
  \end{itemize}
  \centering
   \vspace{0.5cm}
  \
\end{frame}

%--- Ninth Slide: Plotting the results -------------------------------------%
\begin{frame}[fragile]{NDVI variations from 2013 to 2023}
  \textbf{NDVI's difference between 2013 and 2023}
  \vspace{0.5cm}
  \centering
   \vspace{0.5cm}
  \includegraphics[width=1.0\textwidth]{ndvi2013_2018_2023.png}
\end{frame}

%--- Tenth Slide: Plotting an histogram for the NDVI -------------------------------------%
\begin{frame}[fragile]{Creation of an histogram for the distribution of NDVI}
\footnotesize
\begin{lstlisting}
ndvi_rasters <- list("2013" = ndvi2013, "2018" = ndvi2018, 
"2023" = ndvi2023)

rast_to_df <- function(r, anno, tipo){
  df <- as.data.frame(r, xy = FALSE, na.rm = TRUE)
  data.frame(value = df[,1], year = anno, type = tipo)
}

df_ndvi <- bind_rows(
  lapply(names(ndvi_rasters),
         function(y) rast_to_df(ndvi_rasters[[y]], y, "NDVI"))
)
\end{lstlisting}
\end{frame}

%--- Elevnth Slide: Resulting histogram for the NDVI -------------------------------------%

\begin{frame}
\frametitle{NDVI Histogram}
\centering
\includegraphics[width=0.8\textwidth]{histogram_ndvi.png}
\end{frame}

%--- Twelevnth Slide: Plotting a boxplot for the NDVI -------------------------------------%
\begin{frame}[fragile]{Creation of an boxplot for the distribution of NDVI}

\begin{lstlisting}
ggplot(df_ndvi, aes(x = year, y = value, fill = year))
geom_boxplot(outlier.shape = NA, alpha = 0.6) 
facet_wrap(~ type, scales = "free_y") 
labs(title = "Boxplot NDVI by year",
x = "Year", y = "Index value") 
\end{lstlisting}
\end{frame}

%--- Thirteenth Slide: Resulting boxplot for the NDVI ------------------------

\begin{frame}
\frametitle{NDVI Boxplot}
\centering
\includegraphics[width=0.8\textwidth]{boxplot_ndvi.png}
\end{frame}

%--- Fourteenth Slide: Measuring the standard deviation ------------------------
\begin{frame}[fragile]{Measuring Standard deviation 3x3 pixel}

\begin{lstlisting}
sd3nir23 <- focal(nir23, matrix(1/9, 3, 3), fun=sd)
sd3nir13 <- focal(nir13, matrix(1/9, 3, 3), fun=sd)
sd3nir18 <- focal(nir18, matrix(1/9, 3, 3), fun=sd)

par(mfrow=c(1,3))
plot(sd3nir13)
plot(sd3nir18)
plot(sd3nir23)
\end{lstlisting}
\end{frame}

%--- Fifteenth Slide: Measuring the standard deviation -------------------------
\begin{frame}
\frametitle{Standard deviation 3x3}
\centering
\includegraphics[width=1.0\textwidth]{sd_2013_2018_2023.png}
\end{frame}

\end{document}
